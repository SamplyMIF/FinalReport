\chapter{Design}
\label{ch:design}
\section{Intended System Design}
The initial system design is presented here as it was in our initial specification and design, we will highlight any differences or deviation from this in the actual system design section.

\subsection{Expected Components}
\subsubsection{Web Server}
The main interface to the system will be supplied by a web server to allow easy and portable access by members of the MIF. The web facing part of the system will be comprised of several smaller packages all working together under NodeJS:

\begin{enumerate}[label = {}, leftmargin=\widthof{Express-Session |}+\labelsep]
	\item[NodeJS |] A JavaScript runtime that allows execution of JS outside the context of a web browser. 
	
	\item[Express |] An open source web application framework for NodeJS, it is responsible for the high level web server logic and allows a high level interface to web traffic as it handles response codes, etc. itself.
	
	\item[Pug |] An open source high-performance template engine, it allows us to alter the contents of a web page just before serving it to the user, allowing for finer levels of customisation while keeping the code base manageable.
			
	\item[Express-Session |] An add-on for express that gives us easy control of user sessions. This allows us to keep track of user specific data, such as preferred precursors or recently searched compounds.
	
	\item[bCrypt2 |] A widely trusted package for providing user password encryption.
\end{enumerate}

\subsubsection{Computation}
The actual computation will be handled by C++ programs with some small python helper scripts for unit conversion and other pre/post processing. \\

This is done for several reasons, firstly to allow a significant performance increase over JS; secondly to allow us access to widely known mathematics libraries; and thirdly to allow easier modification of the source code by others in the MIF as these are their preferred languages as opposed to JS, this is also the start of an ongoing project which will be maintained and expanded upon after us so maintainability is an important factor. \\

The first step of computation is the stoichiometry calculator. This takes a selection of user defined elements, and then calculate all possible balanced compounds that may be produced by a chemical reaction of these elements. This process also takes into account user limits such as the maximum amount of atoms in the resulting combination to refine the search and prevent the creation of a technically correct but practically unreasonable result. \\

The programs data sources are the web interface which will invoke the users query and the database for relevant information about each element, its output is piped back to NodeJS where it may be cached in the database if it is a common search, the result is then displayed to the user via the web interface. \\

The second stage is the precursor calculator. Its input is a desired ratio of elements, and a set of available chemical precursors. It calculates possible combinations and quantities of the precursors that when mixed together create the desired target ratio. User bounds may also be given to keep the results practically usable.


\subsubsection{Database}
With the theme of future expansion the database will be hosted on a separate server instance running independently of the NodeJS process, this allows for easier expansion of the system if it cannot handle the user load placed on it as both the web server and database can be scaled completely separately from each other without modification to the program code. \\ 

The database will not only store the information needed to run calculations, but will also store the solutions to commonly executed combinations so as to reduce the computational demand by allowing these solutions to be quickly retrieved rather than repeatedly calculated.

\section{Actual System Design}
\subsection{Computation}
\subsubsection{Stoichiometry Calculator}
The first stage of the stoichiometry calculator is to form the $\mathbf{A}$ and $\mathbf{B}$ matrix. As we are always searching for charge balanced compounds, and the result is purely a combination of the inputs with no external factors (e.g. The reaction taking place in air vs under vacuum) we can fix the $\mathbf{B}$ matrix. In this example the user has queried for \underline{Al}uminium, and \underline{O}xygen:

\begin{minipage}{0.4\textwidth}
\vspace{-1em}
\begin{equation*}
	\mathbf{A}=
	\begin{blockarray}{*{2}{c} l}
		\begin{block}{*{2}{>{$\footnotesize}c<{$}} l}
			Al & O  \\
		\end{block}
		\begin{block}{[*{2}{c}]>{$\footnotesize}l<{$}}
			1 & 1 & Initial Quantity \\
			3 & -2 & Charge Imbalance \\
		\end{block}
	\end{blockarray}
\end{equation*}
\begin{equation*}
	\mathbf{B}=
		\begin{blockarray}{*{1}{c} l}
		\begin{block}{[*{1}{c}]>{$\footnotesize}l<{$}}
			1 &  Resulting Proportion \\
			0 &  Desired Charge Imbalance \\
		\end{block}
	\end{blockarray}
\end{equation*}
\begin{equation*}
	\begin{bmatrix}
	1 & 1 \\
	3 & -2
	\end{bmatrix}
	\times
	\begin{bmatrix}
	x_1 \\
	x_2
	\end{bmatrix}
	=
	\begin{bmatrix}
	1 \\
	0 
	\end{bmatrix}
\end{equation*}
\end{minipage}
\hfill
\begin{minipage}{0.6\textwidth}
	\vspace{1.5em}
	This gives us a matrix representation of the constraint equations. However, by our initial design we must consider all possible combinations of oxidation (charge) states in order to find all charge balanced compounds, in this example $O$ has a total of 4 states, and $Al$ a total of 3 giving a total of 132 permutations. While this is true computationally, some of the oxidation states are extremely uncommon and would not be desirable in a physical setting. \\
	
	\vspace{1.5em}
	With this in mind our implementation differs from the design at this point, if we consider all possible charges for\\
\end{minipage}
\vspace{-1.3em}

every element the calculator produces technically feasible but practically impossible results. As this tool is designed to be used by chemists in a real world application we decided to limit the space by removing the rarer charges from the search. However, this data is dynamically loaded we can offer support for rarer charges to be included in the future, however at the current version this is not included. \\

This space reduction leaves only one charge each for $Al$ and $O$, significantly reducing the search space from 132 variants of $\mathbf{A}$ to only one, while still returning all common compounds. \\ 

After having formed the $\mathbf{A}$ matrices, we solve each for a single solution via LU decomposition. Which in it general form takes a matrix $\mathbf{X}$ and finds a lower ($\mathbf{L}$) and upper ($\mathbf{U}$) triangular matrix where 
\begin{equation*}
\mathbf{X} = \mathbf{L}\mathbf{U}
\end{equation*}

As these are triangular matrices we know the first row of $\mathbf{L}$ will take the form:
\begin{equation*}
\begin{bmatrix}
x & 0 & 0 & \dots & 0
\end{bmatrix}
\end{equation*}

With each subsequent row adding another value, left to right, in place of the zeros. By forwards substitution we can obtain a solution in the form:\\
\begin{equation*}
\mathbf{L} \times 
\begin{bmatrix}
y_1 \\
\vdots \\
y_n
\end{bmatrix}
=
\mathbf{B}
\end{equation*}
\vspace{0em}

Calling this solution vector $\mathbf{\hat{y}}$ we can then solve $\mathbf{U}$ with back substitution for our actual solution $\mathbf{\hat{x}}$:
\begin{equation*}
\mathbf{U}\mathbf{\hat{x}} = \mathbf{\hat{y}}
\end{equation*}

This process is not especially computationally intensive, therefore we can solve all required iterations of $\mathbf{A}$ in a relatively short space of time. The result of each iteration is checked for validity (no negative quantities) and given a `score' based on the simplicity of the resulting compound (fewer atoms $\rightarrow$ better score, the actual complexity of the chemistry is ignored as we are computer scientists). After all iterations complete the results are written to a CSV file for the web server to import into the database.

\subsubsection{Precursor Calculator}
Most of the major deviations from the design are within the precursor calculator. We believe this is due to our initial unfamiliarity with the subject field, therefore as our familiarity grew we changed our methods to be both faster, and more effective. \\

The precursor and stoichiometry methods are initially much the same, we must form an $\mathbf{A}$ and $\mathbf{B}$ matrix that properly encode the linear equations we are attempting to solve. In this case we are presuming the input precursor compounds are charged balanced, therefore we can focus solely on the ratios of elements in the desired ratio and the precursor compounds.

\vspace{1.5em}
\begin{minipage}{0.5\textwidth}
	\vspace{-1em}
	\begin{equation*}
		\begin{blockarray}{*{5}{c} l}
		\begin{block}{*{5}{>{$\footnotesize}c<{$}} l}
		$Li_2S$ & $Al_2S_3$ & $Al_2O_3$ & $LiAlO_2$ & $Li_2O$ \\
		\end{block}
		\begin{block}{[*{5}{c}]>{$\footnotesize}l<{$}}
		2 & 0 & 0 & 1 & 2 & $Li$ \\
		0 & 2 & 2 & 1 & 0 & $Al$ \\
		1 & 3 & 0 & 0 & 0 & $S$ \\
		0 & 0 & 3 & 2 & 1 & $O$ \\
		\end{block}
		\end{blockarray}
	\end{equation*}	
\end{minipage}
\hfill
\begin{minipage}{0.45\textwidth}
	The $\mathbf{A}$ matrix encodes the quantity of each element in the precursors, each column represents a precursor and each row a specific element. The $\mathbf{B}$ matrix is simply the desired output stoichiometry in a column vector, with the cosponsoring rows between the two matrices representing the same element. \\
\end{minipage}

We then solve the system of equations via LU decomposition as shown previously, however in this case we are not looking for a single solution as we have a potentially infinite solution space. Therefore we take additional steps to generate more solutions from our first, this is done by first finding the null space of the $\mathbf{A}$ matrix, then diving the null space matrix into a set of vectors which when multiplied by arbitrary real scalars and added to our initial solution produce a new point within the solution space. To continue our stoichiometry example of $Li$, $Al$, $S$, $O$, and taking the $\mathbf{A}$ matrix shown above:

%% TODO: EXAMPLE 

\subsection{Data Structures}
The main source of data is the MongoDB database, and a pair of text files parsed by the calculators. We use the database to store all calculator results allowing us to lighten the computational load of the system and deliver results faster by first querying for stored data and only invoking the calculators if pre-existing results are not found.

The calculators share several classes which are used to store, process, and present data: \\

\texttt{Element}, is used to store all relevant data about each element we are considering in a specific query. \\

\texttt{Reagent}, is a fixed vector of elements representing a precursor in the query, it also contains a string field for the human readable chemical name. \\

\texttt{Solution}, a dynamic vector of doubles, used to store the raw unprocessed output from the algorithm.\\

Each of these classes then has a manager class which helps us access and use them in the wider context of the program: \\

\texttt{SolutionDB}, this maps instances of \texttt{Solution} to unique and predictable strings which act as keys. This allows us to quickly check the uniqueness of a new solution before we accept it into the output list, this is necessary as the generation of new solutions from the initial solution and null space is not guaranteed to be unique. \\

\texttt{ElementDB}, unordered map of element instances and strings which contain the chemical symbol of that particular element, ensures we only load each elements information once, and allows cross referencing between chemical symbols from the user input and in-memory data for the computation itself. \\

\texttt{ReagentDB}, mostly used for processes outside the main bulk of the computation. \\

\texttt{StateDB}, which has no corresponding child class, is a multimap (type of map which allows multiple values for one key) which stores the possible charge state(s) for each element. Keys take the form of the chemical symbol of the element, which is mapped to a set of integers representing possible charges.

%% Work Marker

\subsubsection{Database Structure} 
Database tables are generated on-the-fly as required, however general types of table do exist:

\begin{enumerate}[label = {}, leftmargin=\widthof{CalcedPoints |}+\labelsep]
	\item[UserData |] Usernames, Encrypted Passwords, and other user preference data.
	
	\item[ElementData |] Charge configurations, and other static element data needed for calculations.
	
	\item[CalcedPoints |] Generally a table per invocation of the precursor calculator, storing the results of our calculations so they can be easily referenced in future. 
\end{enumerate}

The exact structure of a CalcedPoints table depends heavily on the query that created it, however given a query its structure will always be predictable: 

\begin{table}[h!]
	\centering
	\begin{tabular}{|c|c|c|}
		\hline
		\multicolumn{3}{|c|}{\textless{}Desired Ratio\textgreater{}} \\ \hline
		\textless{}Precursor 1\textgreater{} & \textless{}Precursor 2\textgreater{} & $\dots$ \\ \hline
		\textless{}Ratio of P1\textgreater{} & \textless{}Ratio of P2\textgreater{} & $\dots$ \\
		$\vdots$ & $\vdots$ & $\ddots$ \\
	\end{tabular}
\end{table}

Each row of the table is a possible solution with the values in each column representing the proportion of that precursor relative to the others.

\subsubsection{System Memory Management}
On the computer which runs the main web server instance availability of RAM may become an issue, this would mainly be caused by multiple large queries executing simultaneously and such a situation is hard to avoid, however the memory footprint of the application is kept to a minimum by invoking the calculators as child processes instead of keeping them with the main NodeJS thread.\\

This allows the memory heavy segment of the system to only take up space in RAM when it is required, and also only occupy the exact amount it requires during computation. If resources are not available for a query to use it will be queued until they are available on a first-come first-served basis. 

\subsubsection{Computation}
The primary data structure in use during the computational stages are matrices, they are used to store the coefficients that represent the system of equations we have generated and will then solve. The software does occasionally cache small amounts of data for its own use, this is not stored in the database and is instead written and read as comma separated values in text files.


\subsection{Interface Design}
The main user interface is provided via the web server, we have opted for a single page style design: the further down the page you go, the more in-depth the data. \\

The top of the page displays a periodic table a user can use to enter elements they would like to pass to the stoichiometry calculator who's result will be displayed below the periodic table as it is the next level of detail.\\

Next the user can select a desired final ratio of elements and a set of precursors they have avalible from the stoichiometry calculator, this will then trigger the precursor calculator who's results will be displayed graphically as either a 2d or 3d triangle/pyramid. This visualisation will be rendered using WebGL in order to allow us to have an interactive 3d environment. \\

The user will then be able to explore possible solutions via the visualisation, selecting a point then a direction to move in, more possible solutions in their desired direction will be generated and displayed allowing them to find a good ratio for their own uses, be this quantities of precursor avalible or ease of measurement.

\subsubsection{User Data}
A user will be able to login if they choose to, accounts are not required. Logging in will offer benefits such as previous searches being remembered and the ability to define preferred precursors which will be displayed separately to the generated precursors to allow easier selection of the precursors they will be using most often.\\

Users accounts data will be protected by bCrypt2 and on disk encryption if possible. Also enforcing a HTTPS connection to the website should ensure password security, as the amount of user provided text is low there is little opportunity for SQL injection, most of the users interaction with the server is via click-able GUI elements which provide a minimal security risk.